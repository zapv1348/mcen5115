\documentclass{article}

\usepackage[margin=1in]{geometry}
\usepackage{graphicx}
\usepackage{amsmath}
\usepackage{amsfonts}
\usepackage{float}
\usepackage{listings}
\usepackage{color}

\definecolor{mygreen}{rgb}{0,0.6,0}
\definecolor{mygray}{rgb}{0.5,0.5,0.5}
\definecolor{mymauve}{rgb}{0.58,0,0.82}
\definecolor{purple}{rgb}{1,0,1}
\definecolor{orange}{rgb}{1,0.6,0}

\lstdefinestyle{customc}{
    belowcaptionskip=1\baselineskip,
    breaklines=true,
    frame=L,
    xleftmargin=\parindent
    language=C,
    showstringspaces=false,
    basicstyle=\footnotesize\ttfamily,
    keywordstyle=\bfseries\color{mygreen}
    commentstyle=\itshape\color{purple}
    identifierstyle=\color{blue},
    stringstyle=\color{orange},
}

\lstset{escapechar=@,style=customc}

\author{Zachary Vogel}
\date{\today}
\title{Notes in MCEN 5115}

\begin{document}
\maketitle


\section*{List of functions}
\subsection*{microcontroller}
\begin{itemize}
    \item init\_colo
        \begin{itemize}
            \item determines color we are looking for
        \end{itemize}
    \item send\_request
        \begin{itemize}
            \item must be able to take in a parameter to determine if we are looking for middle circle or one of the hoops
            \item going to use uart to make the request
            \item will send a code that will correspond to these
        \end{itemize}
    \item receive\_data
        \begin{itemize}
            \item recieves 2 bytes of data, first is the distance to the center of the target, second is angle from center of target.
            \item if all values are zero, need to look for hoop, calls move\_robo with an angle and no position.
        \end{itemize}
    \item move\_robo
        \begin{itemize}
            \item must be able to take in a vector which will define 2-d position from where we are to move to
            \item also need a 1-d angle to figure out which way to point relative to current position
            \item should take into account the motor effects
        \end{itemize}
    \item launch\_balls
        \begin{itemize}
            \item actuate the motor to shoot
            \item should only shoot if we have recieved that we are in the ccorrect position from recieve\_request
        \end{itemize}
    \item pick\_balls
        \begin{itemize}
            \item actuates the fan to pick up balls
        \end{itemize}
    \item check\_lines
        \begin{itemize}
            \item checks the line\_followers to see if we are on the edge of the ball area
        \end{itemize}
\end{itemize}

\subsection*{Odroid}
will be constantly waiting to call receive data
\begin{itemize}
    \item receive\_request
        \begin{itemize}
            \item receives the data available in the uart
            \item processes it to call the correct function
        \end{itemize}
    \item look\_hoop
        \begin{itemize}
            \item looks for the hoop
            \item if it finds it, calculates angle from the center hoop and distance to the center
            \item otherwise returns 0,0
        \end{itemize}
    \item look\_center
        \begin{itemize}
            \item finds the center circle and returns distance to the center of it
        \end{itemize}
    \item send\_data
        \begin{itemize}
            \item will take the data from look\_hoop or look\_center and send it over Uart
        \end{itemize}
\end{itemize}


\section*{Psuedo code}
\subsection*{Microcontroller}
\begin{lstlisting}

\end{lstlisting}

\end{document}
